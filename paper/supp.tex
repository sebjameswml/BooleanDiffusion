% This is an example of using latex for a paper/report of specified
% size/layout. It's useful if you want to provide a PDF that looks
% like it was made in a normal word processor.

% While writing, don't stop for errors
\nonstopmode

% Use the article doc class, with an 11 pt basic font size
\documentclass[11pt, a4paper]{article}

% Makes the main font Nimbus Roman, a Times New Roman lookalike:
%\usepackage{mathptmx}% http://ctan.org/pkg/mathptmx
% OR use this for proper Times New Roman (from msttcorefonts package
% on Ubuntu). Use xelatex instead of pdflatex to compile:
\usepackage{fontspec}
\usepackage{xltxtra}
\usepackage{xunicode}
\defaultfontfeatures{Scale=MatchLowercase,Mapping=tex-text}
\setmainfont{Times New Roman}

% Set margins
\usepackage[margin=2.5cm]{geometry}

% Multilingual support
\usepackage[english]{babel}

% Nice mathematics
\usepackage{amsmath}
\usepackage{esint}

% Left right harpoons for kinetic equations
\usepackage{mathtools}

% Control over maketitle
\usepackage{titling}

% Section styling
\usepackage{titlesec}

% Ability to use colour in text
\usepackage[usenames]{color}

% For the \degree symbol
\usepackage{gensymb}

% Allow includegraphics and nice wrapped figures
\usepackage{graphicx}
\usepackage{wrapfig}
\usepackage[outercaption]{sidecap}

% Set formats using titlesec
\titleformat*{\section}{\bfseries\rmfamily}
\titleformat*{\subsection}{\bfseries\itshape\rmfamily}

% thetitle is the number of the section. This sets the distance from
% the number to the section text.
\titlelabel{\thetitle.\hskip0.3em\relax}

% Set title spacing with titlesec, too.  The first {1.0ex plus .2ex
% minus .7ex} sets the spacing above the section title. The second
% {-1.0ex plus 0.2ex} sets the spacing the section title to the
% paragraph.
\titlespacing{\section}{0pc}{1.0ex plus .2ex minus .7ex}{-1.1ex plus 0.2ex}

%% Trick to define a language alias and permit language = {en} in the .bib file.
% From: http://tex.stackexchange.com/questions/199254/babel-define-language-synonym
\usepackage{letltxmacro}
\LetLtxMacro{\ORIGselectlanguage}{\selectlanguage}
\makeatletter
\DeclareRobustCommand{\selectlanguage}[1]{%
  \@ifundefined{alias@\string#1}
    {\ORIGselectlanguage{#1}}
    {\begingroup\edef\x{\endgroup
       \noexpand\ORIGselectlanguage{\@nameuse{alias@#1}}}\x}%
}
\newcommand{\definelanguagealias}[2]{%
  \@namedef{alias@#1}{#2}%
}
\makeatother
\definelanguagealias{en}{english}
\definelanguagealias{eng}{english}
%% End language alias trick

%% Any aliases here
\newcommand{\mb}[1]{\mathbf{#1}} % this won't work?
% Emphasis and bold.
\newcommand{\e}{\emph}
\newcommand{\mycite}[1]{\cite{#1}}
\newcommand{\code}[1]{\textsf{#1}}
\newcommand{\dvrg}{\nabla\vcdot\nabla}
%% END aliases

% Custom font defs
% fontsize is \fontsize{fontsize}{linespacesize}
\def\authorListFont{\fontsize{11}{11} }
\def\corrAuthorFont{\fontsize{10}{10} }
\def\affiliationListFont{\fontsize{11}{11}\itshape }
\def\titleFont{\fontsize{14}{11} \bfseries }
\def\textFont{\fontsize{11}{11} }
\def\sectionHdrFont{\fontsize{11}{11}\bfseries}
\def\bibFont{\fontsize{10}{10} }
\def\captionFont{\fontsize{10}{10} }

% Caption font size to be small.
\usepackage[font=small,labelfont=bf]{caption}

% Make a dot for the dot product, call it vcdot for 'vector calculus
% dot'. Bigger than \cdot, smaller than \bullet.
\makeatletter
\newcommand*\vcdot{\mathpalette\vcdot@{.35}}
\newcommand*\vcdot@[2]{\mathbin{\vcenter{\hbox{\scalebox{#2}{$\m@th#1\bullet$}}}}}
\makeatother

\def\firstAuthorLast{James}

% Affiliations
\def\Address{\\
\affiliationListFont Adaptive Behaviour Research Group, Department of Psychology,
  The University of Sheffield, Sheffield, UK \\
}

% The Corresponding Author should be marked with an asterisk. Provide
% the exact contact address (this time including street name and city
% zip code) and email of the corresponding author
\def\corrAuthor{Seb James}
\def\corrAddress{Department of Psychology, The University of Sheffield,
  Western Bank, Sheffield, S10 2TP, UK}
\def\corrEmail{seb.james@sheffield.ac.uk}

% Figure out the font for the author list..
\def\Authors{\authorListFont Sebastian James\\[1 ex]  \Address \\
  \corrAuthorFont $^{*}$ Correspondence: \corrEmail}

% No page numbering please
\pagenumbering{gobble}

% A trick to get the bibliography to show up with 1. 2. etc in place
% of [1], [2] etc.:
\makeatletter
\renewcommand\@biblabel[1]{#1.}
\makeatother

% reduce separation between bibliography items if not using natbib:
\let\OLDthebibliography\thebibliography
\renewcommand\thebibliography[1]{
  \OLDthebibliography{#1}
  \setlength{\parskip}{0pt}
  \setlength{\itemsep}{0pt plus 0.3ex}
}

% Set correct font for bibliography (doesn't work yet)
%\renewcommand*{\bibfont}{\bibFont}

% No paragraph indenting to match the VPH format
\setlength{\parindent}{0pt}

% Skip a line after paragraphs
\setlength{\parskip}{0.5\baselineskip}
\onecolumn

% titling definitions
\pretitle{\begin{center}\titleFont}
\posttitle{\par\end{center}\vskip 0em}
\preauthor{ % Fonts are set within \Authors
        \vspace{-1.1cm} % Bring authors up towards title
        \begin{center}
        \begin{tabular}[t]{c}
}
\postauthor{\end{tabular}\par\end{center}}

% Define title, empty date and authors
\title {
  Boolean diffusion: Using Boolean networks as the \emph{reaction} in reaction-diffusion models
}
\date{} % No date please
\author{\Authors}

%% END OF PREAMBLE

\begin{document}

\setlength{\droptitle}{-1.8cm} % move the title up a suitable amount
\maketitle

\vspace{-1.8cm} % HACK bring the introduction up towards the title. It
                % would be better to do this with titling in \maketitle

%%%%%%%%%%%%%%%%%%%%%%%%%%%%%%%%%%%%%%%%%%%%%%%%%%%%%%%%%%%%%%%%%%%%%%%%%%%%%%%
\section{Introduction}

A Boolean network (Kauffman's $NK$-model~\cite{kauffman_origins_1993}) can be
used to represent the genetic machinery in a biological cell. A network of $N$
nodes can be taken to represent $N$ gene/protein actors, which interact with
one another according to a table. To keep the model as simple as possible,
it's assumed that there is a 1-1 relationship between genes and proteins and
that the activation of a gene is equivalent to the production of its
corresponding protein. The activation of a gene may interact with the future
state of the system---its protein product may activate or suppress any of the
$N$ genes. Imagine $N=3$ and the genes are named $a$, $b$ and $c$. Using this
table, it would be possible to look up what the future state of $c$ will be
in, for example, the case that only $a$ is currently expressed or in the case
that any other combination of $a$, $b$ and $c$ are expressed. The table can
define relationships such as ``if proteins $a$ and $b$ are both present, then
protein $c$ will be expressed'' and ``protein $c$ supresses protein
$a$''. Repeatedly updating the states of the three genes according to the
table allows the system to represent a developmental process. Note though,
that we do not know anything about the amount of time that may occur between
updates.

Because the relationships in the table are Boolean, it takes the form of a
truth table, which makes it easy to apply random modifications to the system:
Bits in the truth table can be flipped and the behaviour of the system
examined. As such, it forms a useful, albeit abstract and highly simplified
model to explore how random mutations to a `genome' (the table) affect
developmental processes. If the state of the system can be used to define a
fitness, then evolutionary algorithms can be applied to find out whether
evolution could find a viable solution to a set of environmental constraints.

What is the spatial extent of this system? That is up to the researcher to
decide. It could be applied to a single strand of DNA and the molecules in its
immediate vicinity, across a complete cell, or within a region of tissue. It
is attractive to consider the model as applying to a cell because a within one
cell it seems like a reasonable simplification to suggest that protein
concentrations are uniform. How, then, would processes within one cell
communicate information to other cells? One way is for protein products to
pass through (or otherwise send signals via) the cell membrane, travel through
the extracellular matrix between cells, and into other cells. Such proteins
are called morphogens. Another way is for the cells to reproduce, so that
there are more cells which will generate a particular combination of gene
products.

%%%%%%%%%%%%%%%%%%%%%%%%%%%%%%%%%%%%%%%%%%%%%%%%%%%%%%%%%%%%%%%%%%%%%%%%%%%%%%%
\section{First system}

Average protein densities, $a_i$ for each of $N$ genes operating in the cells
within a spatial element are arbitrary. Pass these through a sigmoid or
threshold, $\sigma()$ to determine whether a gene's input is on or
off. Operate the Boolean net, $G$, and use its output for the $i^{th}$
protein, $G_i$, to add $\Delta_i$ to the protein density. Allow proteins to
diffuse with constant $D_i$ (perhaps choose two representative length scales
for cell-cell interactions and morphogen-type diffusion). Give each protein a
characteristic decay lifetime, $\alpha_i$ (either set
%
$\alpha_i = \alpha,~\forall i$ or, as for diffusion, choose two values for
fast- and slow-decay proteins). Then we can write $N$ differential equations as:

\begin{equation} \label{eq:bd}
\frac{\partial a_i(\mb{x},t)}{\partial t} = D_i \dvrg a_i(\mb{x},t) -\alpha_i
a_i(\mb{x},t) + \Delta_i G_i\big(\sigma(a_1), \sigma(a_2),...,\sigma(a_N)\big)
\end{equation}

Run the system and colour regions according to $a_i$ and also perhaps which
basin of activity $G$ is operating in. That'd be cool.

%
%
BIBLIOGRAPHY
%
\selectlanguage{English}
\bibliographystyle{abbrvnotitle}
% The bibliography NoTremor.bib is the one exported from Zotero. It
% may be necessary to run my UTF-8 cleanup script, bbl_utf8_to_latex.sh
%%\bibliography{NoTremor}
%
% It produces this:
\begin{thebibliography}{1}

\bibitem{kauffman_origins_1993}
Stuart~A.~Kauffman.
\newblock {\em The {Origins} of {Order:} {Self-Organization} and {Selection}
in {Evolution}}.
\newblock New York: Oxford University Press, 1993.

\end{thebibliography}

\end{document}
