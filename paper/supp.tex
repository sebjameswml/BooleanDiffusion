% This is an example of using latex for a paper/report of specified
% size/layout. It's useful if you want to provide a PDF that looks
% like it was made in a normal word processor.

% While writing, don't stop for errors
\nonstopmode

% Use the article doc class, with an 11 pt basic font size
\documentclass[11pt, a4paper]{article}

% Makes the main font Nimbus Roman, a Times New Roman lookalike:
%\usepackage{mathptmx}% http://ctan.org/pkg/mathptmx
% OR use this for proper Times New Roman (from msttcorefonts package
% on Ubuntu). Use xelatex instead of pdflatex to compile:
\usepackage{fontspec}
\usepackage{xltxtra}
\usepackage{xunicode}
\defaultfontfeatures{Scale=MatchLowercase,Mapping=tex-text}
\setmainfont{Times New Roman}

% Set margins
\usepackage[margin=2.5cm]{geometry}

% Multilingual support
\usepackage[english]{babel}

% Nice mathematics
\usepackage{amsmath}
\usepackage{esint}

% Left right harpoons for kinetic equations
\usepackage{mathtools}

% Control over maketitle
\usepackage{titling}

% Section styling
\usepackage{titlesec}

% Ability to use colour in text
\usepackage[usenames]{color}

% For the \degree symbol
\usepackage{gensymb}

% Allow includegraphics and nice wrapped figures
\usepackage{graphicx}
\usepackage{wrapfig}
\usepackage[outercaption]{sidecap}

% Set formats using titlesec
\titleformat*{\section}{\bfseries\rmfamily}
\titleformat*{\subsection}{\bfseries\itshape\rmfamily}

% thetitle is the number of the section. This sets the distance from
% the number to the section text.
\titlelabel{\thetitle.\hskip0.3em\relax}

% Set title spacing with titlesec, too.  The first {1.0ex plus .2ex
% minus .7ex} sets the spacing above the section title. The second
% {-1.0ex plus 0.2ex} sets the spacing the section title to the
% paragraph.
\titlespacing{\section}{0pc}{1.0ex plus .2ex minus .7ex}{-1.1ex plus 0.2ex}

%% Trick to define a language alias and permit language = {en} in the .bib file.
% From: http://tex.stackexchange.com/questions/199254/babel-define-language-synonym
\usepackage{letltxmacro}
\LetLtxMacro{\ORIGselectlanguage}{\selectlanguage}
\makeatletter
\DeclareRobustCommand{\selectlanguage}[1]{%
  \@ifundefined{alias@\string#1}
    {\ORIGselectlanguage{#1}}
    {\begingroup\edef\x{\endgroup
       \noexpand\ORIGselectlanguage{\@nameuse{alias@#1}}}\x}%
}
\newcommand{\definelanguagealias}[2]{%
  \@namedef{alias@#1}{#2}%
}
\makeatother
\definelanguagealias{en}{english}
\definelanguagealias{eng}{english}
%% End language alias trick

%% Any aliases here
\newcommand{\mb}[1]{\mathbf{#1}} % this won't work?
% Emphasis and bold.
\newcommand{\e}{\emph}
\newcommand{\mycite}[1]{\cite{#1}}
\newcommand{\code}[1]{\textsf{#1}}
\newcommand{\dvrg}{\nabla\vcdot\nabla}
%% END aliases

% Custom font defs
% fontsize is \fontsize{fontsize}{linespacesize}
\def\authorListFont{\fontsize{11}{11} }
\def\corrAuthorFont{\fontsize{10}{10} }
\def\affiliationListFont{\fontsize{11}{11}\itshape }
\def\titleFont{\fontsize{14}{11} \bfseries }
\def\textFont{\fontsize{11}{11} }
\def\sectionHdrFont{\fontsize{11}{11}\bfseries}
\def\bibFont{\fontsize{10}{10} }
\def\captionFont{\fontsize{10}{10} }

% Caption font size to be small.
\usepackage[font=small,labelfont=bf]{caption}

% Make a dot for the dot product, call it vcdot for 'vector calculus
% dot'. Bigger than \cdot, smaller than \bullet.
\makeatletter
\newcommand*\vcdot{\mathpalette\vcdot@{.35}}
\newcommand*\vcdot@[2]{\mathbin{\vcenter{\hbox{\scalebox{#2}{$\m@th#1\bullet$}}}}}
\makeatother

\def\firstAuthorLast{James}

% Affiliations
\def\Address{\\
\affiliationListFont Adaptive Behaviour Research Group, Department of Psychology,
  The University of Sheffield, Sheffield, UK \\
}

% The Corresponding Author should be marked with an asterisk. Provide
% the exact contact address (this time including street name and city
% zip code) and email of the corresponding author
\def\corrAuthor{Seb James}
\def\corrAddress{Department of Psychology, The University of Sheffield,
  Western Bank, Sheffield, S10 2TP, UK}
\def\corrEmail{seb.james@sheffield.ac.uk}

% Figure out the font for the author list..
\def\Authors{\authorListFont Sebastian James\\[1 ex]  \Address \\
  \corrAuthorFont $^{*}$ Correspondence: \corrEmail}

% No page numbering please
\pagenumbering{gobble}

% A trick to get the bibliography to show up with 1. 2. etc in place
% of [1], [2] etc.:
\makeatletter
\renewcommand\@biblabel[1]{#1.}
\makeatother

% reduce separation between bibliography items if not using natbib:
\let\OLDthebibliography\thebibliography
\renewcommand\thebibliography[1]{
  \OLDthebibliography{#1}
  \setlength{\parskip}{0pt}
  \setlength{\itemsep}{0pt plus 0.3ex}
}

% Set correct font for bibliography (doesn't work yet)
%\renewcommand*{\bibfont}{\bibFont}

% No paragraph indenting to match the VPH format
\setlength{\parindent}{0pt}

% Skip a line after paragraphs
\setlength{\parskip}{0.5\baselineskip}
\onecolumn

% titling definitions
\pretitle{\begin{center}\titleFont}
\posttitle{\par\end{center}\vskip 0em}
\preauthor{ % Fonts are set within \Authors
        \vspace{-1.1cm} % Bring authors up towards title
        \begin{center}
        \begin{tabular}[t]{c}
}
\postauthor{\end{tabular}\par\end{center}}

% Define title, empty date and authors
\title {
  Boolean diffusion: Using Boolean networks as the \emph{reaction} in reaction-diffusion models
}
\date{} % No date please
\author{\Authors}

%% END OF PREAMBLE

\begin{document}

\setlength{\droptitle}{-1.8cm} % move the title up a suitable amount
\maketitle

\vspace{-1.8cm} % HACK bring the introduction up towards the title. It
                % would be better to do this with titling in \maketitle

%%%%%%%%%%%%%%%%%%%%%%%%%%%%%%%%%%%%%%%%%%%%%%%%%%%%%%%%%%%%%%%%%%%%%%%%%%%%%%%
\section{Introduction}

A Boolean network (Kauffman's $NK$-model~\cite{kauffman_origins_1993}) can be
used to represent the genetic machinery in a biological cell. A network of $N$
nodes can be taken to represent $N$ gene/protein actors, which interact with
one another according to a table. To keep the model as simple as possible,
it's assumed that there is a 1-1 relationship between genes and proteins and
that the activation of a gene is equivalent to the production of its
corresponding protein. The activation of a gene may interact with the future
state of the system---its protein product may activate or suppress any of the
$N$ genes. Imagine $N=3$ and the genes are named $a$, $b$ and $c$. Using this
table, it would be possible to look up what the future state of $c$ will be
in, for example, the case that only $a$ is currently expressed or in the case
that any other combination of $a$, $b$ and $c$ are expressed. The table can
define relationships such as ``if proteins $a$ and $b$ are both present, then
protein $c$ will be expressed'' and ``protein $c$ supresses protein
$a$''. Repeatedly updating the states of the three genes according to the
table allows the system to represent a developmental process. Note though,
that we do not know anything about the amount of time that may occur between
updates.

Because the relationships in the table are Boolean, it takes the form of a
truth table, which makes it easy to apply random modifications to the system:
Bits in the truth table can be flipped and the behaviour of the system
examined. As such, it forms a useful, albeit abstract and highly simplified
model to explore how random mutations to a `genome' (the table) affect
developmental processes. If the state of the system can be used to define a
fitness, then evolutionary algorithms can be applied to find out whether
evolution could find a viable solution to a set of environmental constraints.

What is the spatial extent of this system? That is up to the researcher to
decide. It could be applied to a single strand of DNA and the molecules in its
immediate vicinity, across a complete cell, or within a region of tissue. It
is attractive to consider the model as applying to a cell because a within one
cell it seems like a reasonable simplification to suggest that protein
concentrations are uniform. How, then, would processes within one cell
communicate information to other cells? One way is for protein products to
pass through (or otherwise send signals via) the cell membrane, travel through
the extracellular matrix between cells, and into other cells. Such proteins
are called morphogens. Another way is for the cells to reproduce, so that
there are more cells which will generate a particular combination of gene
products.

%%%%%%%%%%%%%%%%%%%%%%%%%%%%%%%%%%%%%%%%%%%%%%%%%%%%%%%%%%%%%%%%%%%%%%%%%%%%%%%
\section{First system}

The average protein densities, $a_i$ for each of $N$ genes operating in the
cells within a spatial element are continuously valued. Pass these through a
threshold function, $T()$ to (i) determine whether a gene's input to a
gene regulatory network (GRN) is on or off (forming a current state, $s$) and
(ii) obtain a value for the `strength' of that gene's input. Operate the GRN,
$G$, on $s$ and use its output for the $i^{th}$ protein, $G_i$, along with
$T_1,...,T_N$ as inputs to a function $F_i$, which determines how much
to add to the protein density $a_i$ (modulating by a parameter, $\beta_i$). Allow
proteins to diffuse with constant $D_i$ (perhaps choose two representative
length scales for cell-cell interactions and morphogen-type diffusion). Give
each protein a characteristic decay lifetime, $\alpha_i$ (either set
%
$\alpha_i = \alpha,~\forall i$ or, as for diffusion, choose two values for
fast- and slow-decay proteins). Then we can write $N$ differential equations as:

\begin{equation} \label{eq:bd}
\frac{\partial a_i(\mb{x},t)}{\partial t} = D_i \dvrg a_i(\mb{x},t) -\alpha_i
a_i(\mb{x},t) + \beta_i F_i\big(G_i(s), T(a_1),...,T(a_N)\big)
\end{equation}

where $s$ is the current state of gene protein expression and the operation of
the GRN on $s$ gives a new state, $s'$: $s' = G(s)$.

During implementation, I decided that the function $T()$ should
operate something like this:

\begin{equation} \label{eq:sigma}
T(a_i) = a_i - \xi,
\end{equation}

where $\xi$ is a threshold protein density. If $a_i$ exceeds $\xi$, then
element $i$ of the current GRN state, $s$ is considered to be 1. If $a_i$ does
not exceed $\xi$, then element $i$ of $s$ is 0 and $T(a_i)$ has a
negative value. The $N$ elements, $s_i$, of state $s$ can be written:

\begin{equation} \label{eq:s}
s_i = \begin{cases}
      1 & a_i > \xi \\
      0 & a_i \leq \xi
      \end{cases}
\end{equation}
or, equivalently:
\begin{equation} \label{eq:s2}
s_i = \begin{cases}
      1 & T(a_i) > 0 \\
      0 & T(a_i) \leq 0
      \end{cases}
\end{equation}

The existing state $s$ is processed by the GRN, $s' = G(s)$, and so $s' = G(T(a_1),...,T(a_N))$.

Finally, the function $F$ was chosen so that its elements were given by:

\begin{equation} \label{eq:F}
F_i = \begin{cases}
0 & G_i = 0 \\
\sqrt{\frac{1}{N}\sum_j^N T(a_j)^2}  & G_i = 1
\end{cases}
\end{equation}

$F_i$ is a function of $a_1,...,a_N$ and gives the `expressingness' of gene
$i$ at a given location. $F_i$ is a function of space and time because $a_i$
varies with space and time.

\subsection{results}

See code \code{rd\_bool1.h}.

The responses of the various systems were not as exciting as hoped for. I
initialised the system with varous `points of expression' or `humps of
expression' above a background initialisation equal to the expression
threshold, $\xi$. The $a_i$ decayed and diffused according to the selected
parameters $D$ and $\alpha$. Where expression was substantially above
threshold, you see peaks for $T$ and corresponding expression
($F$). However, there seems to be no way to create spatial structure which
is not fairly directly associated with the initial conditions.

\section{Where next}

What additional features could the gene products have? One idea is to
associate the ability to follow gradients in expression of other gene
products. This might be a cell division response to a morphogen, where newly
created cells tend to move in the direction of the signalling gradient. Thus,
a gene product could have N-1 sets of bits which determine whether that gene
product movies down the gradient of gene $i$ or up the gradient of gene
$i$. If both of these bits were high, then they would cancel out, thuse the
most probable state woudl be to not follow gradient of gene $i$. This would
add $2 N (N-1)$ bits to the genome of $N 2^N$ bits. This would allow the
genome to determine spatial structure.

Another thought relates to timeframes. One issue in the model described is
that when we switch into a new state, that state may not persist, and its gene
products may not be produced for long. One idea would be to implement a
timeframe for which gene production would persist, once switched on. Another
would be to implement different 'switch off' thresholds from 'switch on'
thresholds.

I could also include genome bits to switch between short/long diffusion
constants and slow/fast decay, to allow an evolutionary process to modify this
aspect of the system.

\cite{seirinlee_aberrant_2010} point out that gene expression induces significant
problems for Turing like reaction-diffusion patterning.

\cite{gaffney_gene_2006} describes delays in transcription and translation of
tens of minutes (or even hours), which are similar to the timescales of
developmental events. They used 1/decay as the time delay value, which I will
copy.

\cite{monk_oscillatory_2003} gives a specific example of a gene expression
system with delays.

\section{System 2}

At t=0, compute the state of each hex. Now run the simulation forwards,
allowing diffusion and decay to proceed for $1/(\alpha dt)$ timesteps. At this
point, test the state, and determine if the gene production in each hex should
change. In those for which it \emph{should} change, set a time of last changea
to the current time and simulate forwards, checking at each time step if any
hex should undergo a state change.

% BIBLIOGRAPHY
\selectlanguage{English}
\bibliographystyle{abbrvnotitle}
\bibliography{BooleanDiffusion}

\end{document}
